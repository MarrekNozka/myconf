\documentclass{beamer}

\mode<presentation>
{
  \usetheme{default} %Warsaw Madrid AnnArbor CambridgeUS Boadilla
  
 \usecolortheme{beetle} %dove, beetle, lily

  \usefonttheme{structuresmallcapsserif}
 % \usefonttheme{serif}
 
%puntíky u seznamů
  \useinnertheme{rounded}	
% informační linky
  \useoutertheme{infolines} %shadow,split, sidebar, infolines 
 
 \setbeamercolor{normal text}{fg=black,bg=white} % donastavení barev
 \setbeamercolor{titlelike}{fg=gray}
 \setbeamercolor{alerted text}{fg=gray}
 \setbeamercolor{section in toc}{fg=gray}
 \setbeamercolor{description item}{fg=gray}
 \setbeamercolor{block title}{fg=white,bg=gray}
 \setbeamercolor{navigation symbols}{fg=gray}
 
  \setbeamercovered{transparent}  % Položky se postupně odkrývají z šedé
}

\usepackage{czech}%,bookman}
%\usepackage{bookman}
%\usepackage{palatino}


\title{Ukázka možností Beamru}

\subtitle{V dokumentaci toho najdete mnom více}

\author{Marek Nožka}

\institute[VOŠ a SPŠE Olomouc]{Vyšší odborná škola\\ a \\ Střední průmyslová škola elektrotechnická \\ Olomouc}

\date[OpenChange~2006]{25.\,listopadu 2006}

\logo{\large logo prezentace}

% Delete this, if you do not want the table of contents to pop up at
% the beginning of each subsection:
\AtBeginSubsection[]
{
  \begin{frame}<beamer>
    \frametitle{Ukázka možností Beameru}
    \framesubtitle{OpenChange Olomouc 2006}
    \tableofcontents[currentsection,currentsubsection]
  \end{frame}
}

% If you wish to uncover everything in a step-wise fashion, uncomment
% the following command: 
%\beamerdefaultoverlayspecification{<+->}


\begin{document}

\begin{frame}
  \titlepage
\end{frame}

\begin{frame}
  \frametitle{Osnova}	
  \tableofcontents
  % You might wish to add the option [pausesections]
\end{frame}


\section{Ukázka Beamru}
\begin{frame}
  \frametitle{Ukázka Beamru}
  \framesubtitle{\texttt{www.abclinuxu.cz/clanky/navody/beamer-latex-na-prezentace}}
  You can create overlays\dots
  \begin{itemize}
  \item using the \texttt{pause} command:
    \begin{itemize}
    \item
      pause First item.
      \pause
    \item    
      Second item.
    \end{itemize}
  \item
    using overlay specifications:
    \begin{itemize}
    \item<3->
      3 First item.
    \item<4->
      4 Second item.
    \end{itemize}
  \item
    using the general \texttt{uncover} command:
    \begin{enumerate}
      \uncover<5->{\item
        5 First item.}
      \uncover<6->{\item
        6 Second item.}
    \end{enumerate}
  \end{itemize}
\end{frame}


\begin{frame}
  \frametitle{Obrázky s vysvětlivkami}
  \begin{columns}
    \column{.7\textwidth}
    \includegraphics<1->[width=\textwidth]{ukazka-text.png}
    \column{.3\textwidth}
    \only<1>{Můj textový editor}
  \end{columns}
\vspace{5mm}
\uncover<2>{\LARGE $\backslash$uncover}
\only<3>{\LARGE $\backslash$only } tohle vidíme pořád
\end{frame}



\begin{frame}
  \frametitle{Zvýraznění}
  \begin{block}{Prostředí \emph{block}}
    \begin{itemize}
	  \item  Texty je možné \alert{zvýraznit} pomocí $\backslash$alert.
    \end{itemize}
  \end{block}
  \vskip 1cm
  \begin{description}
	  \item[Prostředí description] je \LaTeX ové prostředí definované v Beamru
	  \item[Další položka] bude takto zvírazněna
  \end{description}

  % The following outlook is optional.
  \end{frame}

\begin{frame}
	\begin{lemma}
	toto je lemma
	\end{lemma}
	\begin{example}
	toto je example
	\end{example}
	\begin{proof}
	toto je proof
	\end{proof}
	\begin{Fakt}
	toto je Fakt
	\end{Fakt}
	\begin{problem}
	toto je problem
	\end{problem}
	\begin{Definition}
	toto je Definition
	\end{Definition}
\end{frame}

\end{document}
	======================================
	vim:foldcolumn=4:foldmethod=indent:foldenable
	vim:ft=tex
