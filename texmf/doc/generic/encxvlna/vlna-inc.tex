%% $Id: vlna-inc.tex 202 2008-06-14 18:55:05Z zw $
% Included file, encoding UTF-8, see vlna-enctex.sty

\ifeng

\section{English manual}
This is a manual for ``vlna'' implemented in enc\TeX\ of February 2003. Enc\TeX\ itself and the
macros for plain \TeX\ were developed by Petr Olšák. \LaTeX\ version was derived from it by Zdeněk
Wagner.

\else

\section{\texorpdfstring{Český manuál}{Cesky manual}}
Toto je manuál balíčku "`vlna"' implementovaného pomocí enc\TeX{}u z února~2003. Enc\TeX\ a makra
pro plain \TeX\ vytvořil Petr Olšák. Verzi pro \LaTeX\ z nich odvodil Zdeněk Wagner.

\fi

%%%%%%%%%%%%%%%%%%%%%%%%%%%%%%%%%%%%%%%%%%%%%%%%%%%%%%

\ifeng

\subsection{Purpose}
The purpose of the package is to insert nonbreakable spaces (\verb:~:, in Czech \textit{vlna} or
\textit{vlnka}) after nonsyllabic prepositions and single letter conjuctions directly while \TeX
ing the document. The macros recognise math and verbatim by \TeX\ means. Inserting nonbreakable
spaces by a preprocessor may never be fully reliable because user defined macros and environments
cannot be recognised.

\else

\subsection{\texorpdfstring{Účel}{Ucel}}
Tento balíček slouží ke vkládání nezlomitelných mezer (vlnek) za neslabičné předložky a
jednopísmenné spojky přímo při \TeX ování dokumentu. Makra rozeznávají matematiku a verbatim \TeX
ovými prostředky. Vkládání nezlomitelných mezer preprocesorem nikdy nemůže být naprosto
spolehlivé, protože uživatelsky definovaná makra a prostředí nelze rozpoznat.

\fi

%%%%%%%%%%%%%%%%%%%%%%%%%%%%%%%%%%%%%%%%%%%%%%%%%%%%%%

\ifeng

\subsection{Requirements}
This package requires enc\TeX\ of February 2003. It is often available in the distribution but is
not activated as default. In order to activate it you have to regenerate your formats using the
\texttt{-enc} switch. In some distributions, such as \TL, you have to edit
\texttt{fmtutil.cnf}, possibly by running \texttt{texconfig}. The excerpt from my
\texttt{fmtutil.cnf} in \TL~2007 is shown in Section~\ref{fmtutil}.

In case you do not have enc\TeX\ at all you have to build it yourself from the sources found at
\url{http://math.feld.cvut.cz/olsak/enctex.html}

\else

\subsection{\texorpdfstring{Požadavky}{Pozadavky}}
Tento balíček vyžaduje enc\TeX\ z února 2003. Ten je obvykle v distribucích obsažen, avšak není
aktivován. Pro jeho aktivaci musíte znovu vygenerovat formáty s použitím parametru \texttt{-enc}. V
některých distribucích, např. v \TL, budete muset upravit soubor \texttt{fmtutil.cnf},
pravděpodovně pomocí programu \texttt{texconfig}. Část mého \texttt{fmtutil.cnf} z \TL~2007
je ukázána v kapitole~\ref{fmtutil}.

V případě, že enc\TeX\ nemáte, musíte si jej zkompilovat sami ze zdrojů, které najdete na
\url{http://math.feld.cvut.cz/olsak/enctex.html}

\fi

%%%%%%%%%%%%%%%%%%%%%%%%%%%%%%%%%%%%%%%%%%%%%%%%%%%%%%

\ifeng

\subsection{Package contents}
The package contains:
\begin{description}
\item[encxvlna.pdf] --- this manual
\item[encxvlna.tex, vlna-inc.tex] --- manual sources
\item[encxvlna.tex] --- plain \TeX\ macros
\item[encxvlna.sty] --- \LaTeX\ package file
\item[License.txt] --- license
\end{description}

File \texttt{vlna.tex} is a part of enc\TeX\ but it was previously distributed with \TL\ in a wrong
directory. Moreover, since enc\TeX\ is not enabled as default, the macros might be used without it.
This will generate error messages that might not be understood by beginners. The modified file
\texttt{encxvlna.tex} differs from
the version included in enc\TeX\ just by a descriptive error message.

\else

\edef\pom{---\space\ignorespaces}

\subsection{\texorpdfstring{Obsah balíčku}{Obsah balicku}}
Balíček obsahuje:
\begin{description}
\item[encxvlna.pdf] \pom tento manuál
\item[encxvlna.tex, vlna-inc.tex] \pom zdrojový kód manuálu
\item[encxvlna.tex] \pom makra pro plain \TeX
\item[encxvlna.sty] \pom balíček pro \LaTeX
\item[License.txt] \pom licence
\end{description}

Soubor \texttt{vlna.tex} je součástí enc\TeX{}u, ale původně byl distribuován v \TL\ ve špatném
adresáři. Protože navíc enc\TeX\ není v \TL\ standardně aktivován, mohou být makra použita bez něj.
To způsobí chyby, jimž nemusí začátečníci rozumět. Modifikovaná verze \texttt{encxvlna.tex} se liší od verze obsažené v
enc\TeX{}u pouze více popisnou chybovou zprávou.

\fi

%%%%%%%%%%%%%%%%%%%%%%%%%%%%%%%%%%%%%%%%%%%%%%%%%%%%%%

\ifeng

\subsection{Usage in plain \TeX}
Usage in plain \TeX\ is simple. Just put the following command before your text:

\else

\subsection{\texorpdfstring{Použití}{Pouziti} v plain \TeX u}
Použití v plain \TeX{}u je velmi jednoduché. Vložte před svůj text příkaz:

\fi

%%%%%%%%%%%%%%%%%%%%%%%%%%%%%%%%%%%%%%%%%%%%%%%%%%%%%%

\smallskip
\begin{verbatim}
\input encxvlna
\end{verbatim}

\smallskip

%%%%%%%%%%%%%%%%%%%%%%%%%%%%%%%%%%%%%%%%%%%%%%%%%%%%%%

\ifeng

Remember that some macro definitions may confuse this package. The best location for the above
mentioned command is \textit{after} all definitions but before the text.

\else

Nezapomeňte, že některé definice maker mohou tento balíček zmást. Nejlepší místo pro výše uvedený
příkaz je tedy \textit{za} všemi definicemi, ale před vlastním textem.

\fi

%%%%%%%%%%%%%%%%%%%%%%%%%%%%%%%%%%%%%%%%%%%%%%%%%%%%%%

\ifeng

\subsection{Usage in \LaTeX}
Usage in \LaTeX\ is similarly simple. Put the following command to your preamble:

\else

\subsection{\texorpdfstring{Použití}{Pouziti} v \LaTeX u}
Použití v \LaTeX{}u je stejně jednoduché. Vložte do preambule příkaz:

\fi

%%%%%%%%%%%%%%%%%%%%%%%%%%%%%%%%%%%%%%%%%%%%%%%%%%%%%%

\smallskip
\begin{verbatim}
\usepackage{encxvlna}
\end{verbatim}

\smallskip

%%%%%%%%%%%%%%%%%%%%%%%%%%%%%%%%%%%%%%%%%%%%%%%%%%%%%%

\ifeng

Remember that some packages and macro definitions may contain code which may confuse this package.
The manual for \textsc{hyperref} says that it should be placed as the last package but it would
report weird error messages if it is loaded after \textsc{encxvlna}. The best place for the above
mentioned command is thus just above \verb;\begin{document};.

The \textsc{encxvlna} package itself postpones some commands using \verb;\AtBeginDocument;. If you load
it by this hook, it will be too late and you will see other weird error messages.

\else

Nezapomeňte, že některé balíčky a definice maker mohou obsahovat kód, který tento balíček zmate.
Návod k balíčku \textsc{hyperref} říká, že má být načten jako poslední, ale právě to způsobí výpis
podivných chybových zpráv, pokud bude balíček \textsc{encxvlna} načten dříve. Nejlepší místo pro výše
zmíněný příkaz je tedy přímo nad \verb;\begin{document};.

Samotný balíček \textsc{encxvlna} využívá \verb;\AtBeginDocument; k odložení některých příkazů. Pokud
se pokusíte využít téhož mechanismu k načtení tohoto balíčku, bude to příliš pozdě a dočkáte se
dalších podivných chybových zpráv.

\fi

%%%%%%%%%%%%%%%%%%%%%%%%%%%%%%%%%%%%%%%%%%%%%%%%%%%%%%

\ifeng

\subsection{Modifications in the \LaTeX\ version}
As already written the \LaTeX\ package was derived from the original plain \TeX\ file. In addition
to creation of the package signature the following modifications were made:

\begin{enumerate}
\item Definition of \verb;\uv; was removed because it appears in the Czech and Slovak language
definition files.
\item Register \verb;\mubytein; is set to 2 at the beginning of the document.
\item Czech and Slovak language definition files for \textsc{babel} introduce in version~3.1
new syntax for writing quotes, namely \verb;"`v lese"';. This is added to the list of recognised
patterns.
\item Definition of \verb;\protect; is tested so that the macros do nothing in moving arguments.
\item Definition of \verb;\rm; is compared to \verb;\@empty; so that outlines are correctly created
by \textsc{hyperref}.
\item Tests were added in order to enable work with the \textsc{microtype} package.
\end{enumerate}

The macros now need 15 \verb;\expandafter;'s!

\else

\subsection{Modifikace ve verzi pro \LaTeX}
Jak již bylo napsáno, verze pro \LaTeX\ byla odvozena z původních maker pro plain \TeX. Kromě
vytvoření signatury \LaTeX ového balíčku byly provedeny tyto změny:

\begin{enumerate}
\item Byla odstraněna definice \verb;\uv;, protože se vyskytuje v jazykových definičních souborech
pro češtinu a slovenštinu.
\item Registr \verb;\mubytein; je naplněn hodnotou 2 až na začátku dokumentu.
\item České a slovenské jazykové definiční soubory pro \textsc{babel} zavádějí od verze~3.1 novou
syntaxi pro zápis uvozovek: \verb;"`v lese"';. Toto je přidáno do seznamu rozpoznávaných vzorů.
\item Testuje se definice \verb;\protect; tak, aby makra nedělala nic v pohyblivých (moving)
argumentech.
\item Definice \verb;\rm; se porovnává s \verb;\@empty;, aby balíček \textsc{hyperref} správně
vytvořil záložky.
\item Byl přidán test, který umožní spolupráci s balíčkem \textsc{microtype}.
\end{enumerate}

Makra nyní potřebují 15 příkazů \verb;\expandafter;!

\fi

%%%%%%%%%%%%%%%%%%%%%%%%%%%%%%%%%%%%%%%%%%%%%%%%%%%%%%

\ifeng

\subsection{Important note for \LaTeX\ users}
There is a problem with the \textsc{url} package and hence with \textsc{hyperref} which loads
\textsc{url}. As a matter of fact it is not a problem of \textsc{encxvlna} but a problem of enc\TeX\
and \textsc{Unicode}. Lines 39 and 50 of \texttt{url.sty} contain unprintable characters which are
not legal as UTF-8 character sequences. Everything works fine until you try to typeset a URL while
converting the input from UTF-8 by enc\TeX. It is sufficient to set \verb;\mubytein; to zero just
before loading \textsc{url} or \textsc{hyperref}. If you load these packages including
\textsc{encxvlna} in correct order just
before \verb;\begin{document};, you need not bother with setting back a nonzero value to
\verb;\mubytein;. Look into the source code of this manual, into file
\else

\subsection{\texorpdfstring{Důležitá poznámka pro uživatele}{Dulezita poznamka pro uzivatele}
\LaTeX u}
Při použití balíčku \textsc{url}, a tudíž \textsc{hyperref}, jenž \textsc{url} načítá, nastává
problém. Ve skutečnosti to není problém balíčku \textsc{encxvlna}, ale problém enc\TeX{}u a
\textsc{Unicode}. Řádky 30 a 50 v souboru \texttt{url.sty} obsahují netisknutelné znaky, jež jsou
nepovolenými znakovými sekvencemi v UTF-8. Vše funguje správně až do chvíle, kdy chcete vytisknout
URL a současně konvertovat vstup z UTF-8 enc\TeX em. Postačí, když vynulujete \verb;\mubytein;
před načtením \textsc{url} či \textsc{hyperref}. Načítáte-li tyto balíčky včetně balíčku
\textsc{encxvlna} ve správném pořadí bezprostředně před příkazem \verb;\begin{document};, nemusíte se starat o
návrat nenulové hodnoty do registru \verb;\mubytein;. Podívejte se do zdrojového kódu tohoto
manuálu, do souboru
\fi
\texttt{encxvlna.tex}
\ifeng
in directory
\else
v adresáři
\fi
\texttt{doc/generic/encxvlna}.

%%%%%%%%%%%%%%%%%%%%%%%%%%%%%%%%%%%%%%%%%%%%%%%%%%%%%%

\ifeng
Similar problems are encountered when using the \textsc{movie15} package. You have to include the
\cmd{includemovie} command within a group setting 
\else
Podobné problémy nastávají při použití balíčku \textsc{movie15}. Musíte příkaz \cmd{includemovie} uzavřít do
skupiny,v níž nastavíte
\fi
\verb;\mubytein=0;.

%%%%%%%%%%%%%%%%%%%%%%%%%%%%%%%%%%%%%%%%%%%%%%%%%%%%%%

\ifeng
\subsection{License}
The package can be used and distributed according to the LaTeX Project Public License version~1.3 or later the
text of which can be found at the \texttt{License.txt} file in the \texttt{doc} directory or at
\else

\subsection{Licence}
Balíček může být používán a šířen podle LaTeX Project Public License verze~1.3 nebo novější, jejíž text najdete
v souboru \texttt{License.txt} v adresáři \texttt{doc}, nebo na
\fi
\url{http://www.latex-project.org/lppl.txt}
