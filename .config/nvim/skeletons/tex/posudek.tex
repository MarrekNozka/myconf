\input utf8-t1
\documentclass[12pt,a4paper]{article}
\usepackage[czech]{babel}
\usepackage[T1]{fontenc}
\usepackage{encxvlna}
\usepackage{spseol}
\usepackage{graphicx} %,graphics  

%\usepackage[active]{srcltx} \include{srctex}


%%% Radkovani  %%%
%\def\baselinestretch{1.25}\normalsize % Radkovani

%%%%%%%%%%%%%%%% OKRAJE %%%%%%%%%%%%%%%%%%%%%%%%ř
\bezPaticky
\begin{document}

\par\vspace*{6mm}\par
{\hspace{\fill}\LARGE \bf Posudek oponenta absolventské práce\hspace{\fill}}
\par\vspace*{7mm}\par

\begin{tabbing}
    \textbf{Název práce:}~~~~~~~\=Realizace digitálního filtru na 8-bitovém procesoru\\
    \textbf{Zpracovatel}: \> Michal Stariat \\
    \textbf{Oponent práce}: \> Ing.\,Marek Nožka \\
    \textbf{Školní rok}: \> 2013/2014 \\
    \hr
\end{tabbing}

Zpracovatel měl za úkol navrhnout jednoduchý číslicový filter a realizovat ho
pomocí 8-bitového procesoru. Konstatuji, že zadání práce je v hrubých rysech 
splněno, obsahuje ale vážné nedostatky.

\par\vspace*{8mm}\par

\noindent \textbf{Návrh hodnocení}: \textsc{Dobře}

\par\vspace*{8mm}\par

\noindent V Olomouci, dne \datum{30}{5}{2014} 
\par\vspace*{0mm}\par\podpis{Ing.\,Marek Nožka}




\end{document}




